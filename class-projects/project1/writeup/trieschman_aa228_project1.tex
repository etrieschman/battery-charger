\documentclass[twoside,11pt]{article}

\usepackage{aa228-jmlr2e}
\usepackage{lipsum}
\usepackage{listings}

\input{julia_listing}

\begin{document}

% Refer to this link for project rubric: https://web.stanford.edu/class/aa228/cgi-bin/wp/project-1/
\title{Project 1: Bayesian Structure Learning}

%===========================================
%===========================================
\name{Erich Trieschman}
\email{etriesch@stanford.edu}


\maketitle


\section{Algorithm Description}
%===========================================
% TODO: Replace this with a short description of your algorithm(s) used.
\lipsum[2]
%===========================================



\section{Graphs}
%===========================================
% TODO: Add your small, medium, and large graph visualizations here
%===========================================
\begin{figure}[h]
    \centering
    \includegraphics[width=0.4\textwidth]{example_graph.pdf}
    \caption{Graph caption.}
\end{figure}



\section{Code}
%===========================================
% TODO: Add your code here, see code listing options here: https://www.overleaf.com/learn/latex/code_listing
% NOTE: Code does not count towards your page limit!
% OPTIONS:
%   1. Paste everything into a {verbatim} environment (where all characters are parsed...verbatim).
%   or 2. paste everything into a {lstlisting} environment for syntax highlighting (examples for Julia and Python below).
% NOTE: Feel free to break up functions into separate {algorithm} + {lstlisting} environments for better organization (not required!) 
%===========================================


%===========================================
% EXAMPLE PYTHON, TODO REPLACE WITH YOUR CODE:
%===========================================
\begin{algorithm}
\begin{lstlisting}[language=Python]
import sys

import networkx


def write_gph(dag, idx2names, filename):
    with open(filename, 'w') as f:
        for edge in dag.edges():
            f.write("{}, {}\n".format(idx2names[edge[0]], idx2names[edge[1]]))


def compute(infile, outfile):
    # WRITE YOUR CODE HERE
    # FEEL FREE TO CHANGE ANYTHING ANYWHERE IN THE CODE
    # THIS INCLUDES CHANGING THE FUNCTION NAMES, MAKING THE CODE MODULAR, BASICALLY ANYTHING
    pass


def main():
    if len(sys.argv) != 3:
        raise Exception("usage: python project1.py <infile>.csv <outfile>.gph")

    inputfilename = sys.argv[1]
    outputfilename = sys.argv[2]
    compute(inputfilename, outputfilename)


if __name__ == '__main__':
    main()

\end{lstlisting}
\end{algorithm}


\end{document}